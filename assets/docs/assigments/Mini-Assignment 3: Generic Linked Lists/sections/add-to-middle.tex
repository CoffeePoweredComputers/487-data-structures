\begin{minipage}{\textwidth}
    \centering
    \begin{tikzpicture}[list/.style={rectangle split, rectangle split parts=2, draw, rectangle split horizontal}, >=stealth, start chain]

        \node[list,on chain] (A) {12};
        \node[list,on chain] (B) {99};
        \node[list,on chain] (C) {37};
        \node[list,on chain] (D) {44};
        \node[list,on chain] (E) {26};

        \node[list, below of=C] (New) {22};
        \node[draw,inner sep=6pt, right of=New] (N1) {};
        \draw (N1.north east) -- (N1.south west);
        \draw (N1.north west) -- (N1.south east);
        \draw[*->] let \p1 = (New.two), \p2 = (New.center) in (\x1,\y2) -- (N1);

        \node[above of=A] (H) {head};
        \node[above of=E] (T) {tail};

        \node[above of=C] (Curr) {curr};
        \node[above of=B] (Prev) {prev};

        \node[on chain,draw,inner sep=6pt] (N) {};
        \draw (N.north east) -- (N.south west);
        \draw (N.north west) -- (N.south east);

        \draw[*->] let \p1 = (A.two), \p2 = (A.center) in (\x1,\y2) -- (B);
        \draw[*->] let \p1 = (B.two), \p2 = (B.center) in (\x1,\y2) -- (C);
        \draw[*->] let \p1 = (C.two), \p2 = (C.center) in (\x1,\y2) -- (D);
        \draw[*->] let \p1 = (D.two), \p2 = (D.center) in (\x1,\y2) -- (E);
        \draw[*->] let \p1 = (E.two), \p2 = (E.center) in (\x1,\y2) -- (N);

        \draw[->] (Curr) -- (C);
        \draw[->] (Prev) -- (B);

        \draw[->] (H) -- (A);
        \draw[->] (T) -- (E);

    \end{tikzpicture}\\
    \textbf{Step 1: Search for and find a reference to the node you want to remove and the node that precedes it.}
\end{minipage}
\begin{minipage}{\textwidth}
    \centering
    \begin{tikzpicture}[list/.style={rectangle split, rectangle split parts=2, draw, rectangle split horizontal}, >=stealth, start chain]

        \node[list,on chain] (A) {12};
        \node[list,on chain] (B) {99};
        \node[list,on chain] (C) {37};
        \node[list,on chain] (D) {44};
        \node[list,on chain] (E) {26};

        \node[list, below of=C] (New) {22};
        \node[draw,inner sep=6pt, right of=New] (N1) {};
        \draw (N1.north east) -- (N1.south west);
        \draw (N1.north west) -- (N1.south east);
        \draw[*->] let \p1 = (New.two), \p2 = (New.center) in (\x1,\y2) -- (N1);

        \node[above of=A] (H) {head};
        \node[above of=E] (T) {tail};

        \node[above of=C] (Curr) {curr};
        \node[above of=B] (Prev) {prev};

        \node[on chain,draw,inner sep=6pt] (N) {};
        \draw (N.north east) -- (N.south west);
        \draw (N.north west) -- (N.south east);

        \draw[*->] let \p1 = (A.two), \p2 = (A.center) in (\x1,\y2) -- (B);
        \draw[*->] let \p1 = (B.two), \p2 = (B.center) in (\x1,\y2) -- (New);
        \draw[*->] let \p1 = (C.two), \p2 = (C.center) in (\x1,\y2) -- (D);
        \draw[*->] let \p1 = (D.two), \p2 = (D.center) in (\x1,\y2) -- (E);
        \draw[*->] let \p1 = (E.two), \p2 = (E.center) in (\x1,\y2) -- (N);

        \draw[->] (Curr) -- (C);
        \draw[->] (Prev) -- (B);

        \draw[->] (H) -- (A);
        \draw[->] (T) -- (E);

    \end{tikzpicture}\\
    \textbf{Step 2: Update previous node's next node to be a reference to the new node.}
\end{minipage}\\
\vspace{0.25cm}\\
\begin{minipage}{\textwidth}
    \centering
    \begin{tikzpicture}[list/.style={rectangle split, rectangle split parts=2, draw, rectangle split horizontal}, >=stealth, start chain]

        \node[list,on chain] (A) {12};
        \node[list,on chain] (B) {99};
        \node[list,on chain] (C) {37};
        \node[list,on chain] (D) {44};
        \node[list,on chain] (E) {26};

        \node[list, below of=C] (New) {22};

        \node[above of=A] (H) {head};
        \node[above of=E] (T) {tail};

        \node[above of=C] (Curr) {curr};
        \node[above of=B] (Prev) {prev};

        \node[on chain,draw,inner sep=6pt] (N) {};
        \draw (N.north east) -- (N.south west);
        \draw (N.north west) -- (N.south east);

        \draw[*->] let \p1 = (A.two), \p2 = (A.center) in (\x1,\y2) -- (B);
        \draw[*->] let \p1 = (B.two), \p2 = (B.center) in (\x1,\y2) -- (New);
        \draw[*->] let \p1 = (New.two), \p2 = (New.center) in (\x1,\y2) -- (C);
        \draw[*->] let \p1 = (C.two), \p2 = (C.center) in (\x1,\y2) -- (D);
        \draw[*->] let \p1 = (D.two), \p2 = (D.center) in (\x1,\y2) -- (E);
        \draw[*->] let \p1 = (E.two), \p2 = (E.center) in (\x1,\y2) -- (N);

        \draw[->] (Curr) -- (C);
        \draw[->] (Prev) -- (B);

        \draw[->] (H) -- (A);
        \draw[->] (T) -- (E);


    \end{tikzpicture}\\
    \textbf{Step 3: Update the new nodes net node to be a reference to the current node}
\end{minipage}



