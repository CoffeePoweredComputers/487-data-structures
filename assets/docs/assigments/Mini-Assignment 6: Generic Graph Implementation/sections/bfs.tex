\begin{minipage}{0.49\textwidth}
\RestyleAlgo{ruled} 
\begin{algorithm}[H]
    \caption{Breadth First Search}\label{alg:bfs}
    \DontPrintSemicolon
    \SetKwFunction{FBFS}{BFS}
    \SetKwProg{Fn}{Function}{}{}
    \Fn{\FBFS{G, Source}}{
        \For{ u $\in$ G.Vertex}{
            u.visited $\gets$ null\;
        }
        Q $\gets \emptyset$\; 
        Q.enqueue(Source)\;
        \While{Q $\neq \emptyset$}{
            u $\gets$ Q.dequeue()\;
            \If{u is visited}{
                continue\;
            }
            u.visited $\gets$ true\;
            \For{edge $\in$ G.Adj[u]}{
                \If{edge.dest is not visited}{
                    Q.enqueue(edge.dest)\;
                }
            }
        }
    }
\end{algorithm}
\textit{Note: the traversals shown on the right will vary depending on the order in which vertecies are pushed to the queue.}
\end{minipage}
\hfill
% NOTE: Credit for this goes to -> https://texample.net/tikz/examples/prims-algorithm/
\begin{minipage}{0.46\textwidth}
\textbf{Results for Directed Graph:}
\begin{figure}[H]
\centering
\begin{tikzpicture}[scale=1.5]

    \node[starting vertex] (a) at (0, 1) {a};
    \foreach \pos/\name in {{(2,1)/b}, {(4,1)/c},
                            {(0,0)/d}, {(3,0)/e}, {(2,-1)/f}, {(4,-1)/g}}
        \node[vertex] (\name) at \pos {$\name$};

    \foreach \source/ \dest in {a/b/7, c/b/8,a/d/5,d/b/9,
                                         e/b/7, e/c/5,e/d/15,
                                         f/d/6,f/e/8,
                                         g/e/9,g/f/11}
        \draw[-{>[scale=4.5]},line width=0.9pt] (\source) to (\dest);
\end{tikzpicture}
\end{figure}
\footnotesize{\textbf{Traversal:}} a b d\\
\hline
\vspace{0.2cm}
\textbf{Results for Undirected Graph:}
\begin{figure}[H]
\centering
\begin{tikzpicture}[scale=1.5, auto,swap]

    \node[starting vertex] (a) at (0, 1) {a};
    \foreach \pos/\name in {{(2,1)/b}, {(4,1)/c},
                            {(0,0)/d}, {(3,0)/e}, {(2,-1)/f}, {(4,-1)/g}}
        \node[vertex] (\name) at \pos {$\name$};

    \foreach \source/ \dest in {b/a, c/b,d/a,d/b,
                                         e/b,e/c,e/d,
                                         f/d,f/e,
                                         g/e,g/f}
        \draw[->, edge] (\source) -- (\dest);

\end{tikzpicture}
\end{figure}
\footnotesize{\textbf{Traversal:}} a b d e c f g
\end{minipage}

\vspace{0.25cm}\\

\noindent Algorithm~\ref{alg:bfs} presents the iterative approach you will take
for implementing BFS from a given source node. The purpose of the variables 
used in that psudeocode are as follows:
\begin{itemize}
    \item \lstinline|G| is the graph on which the alorithm will operate.
    \item \lstinline|Source| is our starting vertex
    \item \lstinline|Q| is a \textit{queue} that contains the ondes under consideration for traversal. We initiallize it to empty, add the starting vertex, and then begin the traversal.
\end{itemize}

As with the other psudeocode we've used in this class, it is important to note
that the psudeocode is simply a general set of instructions on how the solution
should be structured. Your implementation should follow it in spirit but
\textit{will not be a direct translation of the psudeocode to Java}. Noteably,
your vertecies will not have a \lstinline|visited| attribute. You will instead
need to use some other inbuilt data structure (e.g., \lstinline|Map|) to keep
track of whether or not a node has been visited.

