\begin{figure}[H]
    \centering
    \begin{tikzpicture}[list/.style={rectangle split, rectangle split parts=2, draw, rectangle split verical}, start chain]

            \node[list,on chain] (A) {value=12 \nodepart{second} weight=1};

    \end{tikzpicture}\\
    \caption{A visual representation of an edge}
    \label{fig:edge}
\end{figure}

Our implementation of an adjacency list will be composed of a HashMap with
vertices as keys and edge lists as values. As you will see in
Section~\ref{sec:edge}, an edge is composed of two attributes, a weight and the
vertex containing a destination (Figure~\ref{fig:edge}). The following is an
example of how a graph is represented by an adjacency matrix.\\

\begin{minipage}{0.4\textwidth}
    {\renewcommand{\arraystretch}{1.2}% for the vertical padding}
    \begin{tabular}{| c | l |}
        \hline
        Key & Value \\
        \hline
        a & \rule[11.2ex]{0pt}{0pt}  \begin{tikzpicture}[list/.style={rectangle split, rectangle split parts=3, draw}, start chain]

                \foreach \x/ \w in {D/5}{
                    \node[list,on chain] (\x) {dest=\x \nodepart{second} weight=\w};
                }


                \node[on chain,draw,inner sep=6pt] (N) {};
                \draw (N.north east) -- (N.south west);
                \draw (N.north west) -- (N.south east);


                \foreach \s/ \d in {D/N}{
                    \draw[*->] let \p1 = (\s.three), \p2 = (\s.three) in (\x1,\y2) -- (\d);
                }

        \end{tikzpicture}
        \\ \hline
        b & \rule[11.2ex]{0pt}{0pt} \begin{tikzpicture}[list/.style={rectangle split, rectangle split parts=3, draw}, start chain]

                \foreach \x/ \w in {A/7}{
                    \node[list,on chain] (\x) {dest=\x \nodepart{second} weight=\w};
                }


                \node[on chain,draw,inner sep=6pt] (N) {};
                \draw (N.north east) -- (N.south west);
                \draw (N.north west) -- (N.south east);


                \foreach \s/ \d in {A/N}{
                    \draw[*->] let \p1 = (\s.three), \p2 = (\s.three) in (\x1,\y2) -- (\d);
                }

        \end{tikzpicture}
        \\ \hline
        c & \rule[11.2ex]{0pt}{0pt} \begin{tikzpicture}[list/.style={rectangle split, rectangle split parts=3, draw}, start chain]

                \foreach \x/ \w in {B/8}{
                    \node[list,on chain] (\x) {dest=\x \nodepart{second} weight=\w};
                }


                \node[on chain,draw,inner sep=6pt] (N) {};
                \draw (N.north east) -- (N.south west);
                \draw (N.north west) -- (N.south east);


                \foreach \s/ \d in {B/N}{
                    \draw[*->] let \p1 = (\s.three), \p2 = (\s.three) in (\x1,\y2) -- (\d);
                }
        \end{tikzpicture}
        \\ \hline
        d & \rule[11.2ex]{0pt}{0pt} \begin{tikzpicture}[list/.style={rectangle split, rectangle split parts=3, draw}, start chain]

                \foreach \x/ \w in {B/1}{
                    \node[list,on chain] (\x) {dest=\x \nodepart{second} weight=\w};
                }


                \node[on chain,draw,inner sep=6pt] (N) {};
                \draw (N.north east) -- (N.south west);
                \draw (N.north west) -- (N.south east);


                \foreach \s/ \d in {B/N}{
                    \draw[*->] let \p1 = (\s.three), \p2 = (\s.three) in (\x1,\y2) -- (\d);
                }

        \end{tikzpicture}
        \\ \hline
        e & \rule[11.2ex]{0pt}{0pt} \begin{tikzpicture}[list/.style={rectangle split, rectangle split parts=3, draw}, start chain]

                \foreach \x/ \w in {D/15, B/7, C/5}{
                    \node[list,on chain] (\x) {dest=\x \nodepart{second} weight=\w};
                }


                \node[on chain,draw,inner sep=6pt] (N) {};
                \draw (N.north east) -- (N.south west);
                \draw (N.north west) -- (N.south east);


                \foreach \s/ \d in {D/B, B/C, C/N}{
                    \draw[*->] let \p1 = (\s.three), \p2 = (\s.three) in (\x1,\y2) -- (\d);
                }



        \end{tikzpicture}
        \\ \hline

    \end{tabular}
    }
\end{minipage}
\hfill
\begin{minipage}{0.4\textwidth}
\begin{figure}[H]
\centering
\begin{tikzpicture}[scale=0.7, auto,swap]

    \node[starting vertex] (a) at (-5,2) {a};
    \foreach \pos/\name in {{(-2,5)/b}, {(2,2)/c},
                            {(-3,-2)/d}, {(0,0)/e}}
        \node[vertex] (\name) at \pos {$\name$};

    \foreach \source/ \dest/ \weight in {b/a/7, c/b/8,a/d/5,
                                         e/b/7, e/c/5,e/d/15,
                                         d/b/1
                                         }
        \draw[-{>},line width=0.9pt] (\source) -- node[weight] {$\weight$} (\dest);
\end{tikzpicture}
\end{figure}
\end{minipage}

\\
\vspace{0.25cm}
