\subsubsection{Attributes}

\begin{enumerate}
    \item \lstinline|private final boolean directed| 
    \item \lstinline|private final boolean weighted|
    \item \lstinline|private Map<E, List<Edge<E>>> map|
\end{enumerate}

The purpose of the \lstinline|directed| and \lstinline|weighted| is to direct
the control flow of methods relating to the addition and removal of edges. 
The map is the data-structure we will be using to represent our adacency list.
Each vertex \lstinline|E| will be associated with a \lstinline|List| of 
\lstinline|Edge| instances. 

\subsubsection{Constructor}

\paragraph{\lstinline|public ListGraph(boolean directed, boolean weighted) \{
... \}|}: This is the only constructor we will have for this class since
control flow of variations on the graph will be handled via the two parameters.
It should set the \lstinline|directed| and \lstinline|weighted| attributes
equal to the values passed in as parameter s to the constructor. It should also
create a new, empty instance of \lstinline|Map| that associates vertecies of
\lstinline|E| with and edge list of \lstinline|List<Edge<E>>|.



\subsubsection{Vertex Methods}

\paragraph{\lstinline|public void addVertex(E vertex) { ... }|}: The add
vertex method should create a new entry in the \lstinline|map| and associate it
with a new instance of an empty \lstinline|List| of \lstinline|Edges|.

\paragraph{\lstinline|public void removeVertex(E vertex) { ... }|}: The 
remove vertex function should do two things: (1) remove the given vertex
and it's associated list of edges and (2) search through all other 
edge lists in \lstinline|map| and all those edges with the given 
vertex as the \lstinline|dest|.

%\noindent \textit{Hint:} Look into the
%\href{https://docs.oracle.com/javase/8/docs/api/java/util/Collection.html#removeIf-java.util.function.Predicate-}{\lstinline|removeIf|} 
%function for the \lstinline|List| interface in the Java documentation. It can
%be used to reduce the complexity of the second objective of the
%\lstinline|removeVertex| function.

\paragraph{\lstinline|public Set<E> getVertecies() { ... }|}: This 
should return a list of the vertecies currently in \lstinline|map|. \\

\noindent\textit{Hint:} This method should be one line of code that returns
the keyset from the map. Refer to the \lstinline|HashMap| documentation for
which method can accomplish this.

\subsubsection{Edge Methods}

The following two methods will handle the addition of edges to the graph. 
\begin{enumerate}
    \item \lstinline|public void addEdgeUnweighted(E source, E dest) { ... }|
    \item \lstinline|public void addEdgeWeighted(E source, E dest, int weight) { ... }|
\end{enumerate}
Each of these methods should first check if both the given \lstinline|source|
and the \lstinline|dest| should be in the \lstinline|map|. If they are, it
should instantiate a new edge, \textit{taking care to call the appropriate
constructor}, and add the edge to \lstinline|source|'s associate edge list in
\lstinline|map|. If the graph is undirected, it should follow a similar process
to create an edge from \lstinline|dest| to \lstinline|source|.


\paragraph{\lstinline|public void removeEdge(E source, E dest) \{ ... \}|}: 
This method should remove the edge from \lstinline|source| to \lstinline|dest|
and, if the graph graph is undirected, from \lstinline|dest| to \lstinline|source|.

\paragraph{\lstinline|public List<Edge<E>> getEdges(E vertex) \{ ... \}|}: This
is a getter method that retrieves the edgelist associated with a given vertex.\\

\noindent\textit{Hint:} This method should be one line of code that returns the
list associated with a vertex. Refer to the \lstinline|HashMap| documentation
for which method can accomplish this.

