To begin with, we will make a \lstinline|Vehicle| class which will contain
information on each of the vehicles we will eventually read in from our XML
file. 


\subsubsection{Attributes}
The first thing we will need for our class are some attributes to store
information on the vehicles that will eventually be instantiated. Reference
Section \ref{ref:xmlstuff} and create attributes for each of the vehicle
elements we will be storing information on. Be sure to (1) make each attribute
is of the correct type and (2) all attributes are \lstinline|private| and
\lstinline|final|. The purpose of \lstinline|private| is to ensure
encapsulation and \lstinline|final| is used to ensure that, once set, the
values associated with each attribute cannot be changed.\\

\subsubsection{Constructor}
Our constructor should have the following general form:
\begin{lstlisting}[frame=trBL]
public Vehicle(...){
    // Body here
}
\end{lstlisting}
\textbf{Your Task: } Complete the constructor such that it takes parameters for
each of the vehicles attributes and sets them. Refer to the attributes in section
2.2 for which attributes should be included in the constructor and what their
type should be.\\

\subsubsection{Getters}\label{sec:getter}

Since our attributes were declared with the \lstinline|private| keyword they
will not be accessible outside of the class. This keyword allows us to control
access to attributes by defining ``getter'' functions to allow instances of our
class, as they're used in other classes, access to private variables.
Additionally, we can allow other classes to modify the values of private
variables using ``setter'' functions.\\

For our Vehicle class we will only be needing to define getter functions since
all of our attributes are \lstinline|final| and therefore cannot be modified
after their initial assignment.  In general, getters function names should be
the word ``get'' followed by the name of the attribute we want to get. For
instance, if your class has a private attribute string attribute named
\lstinline|studentName| the associated getter function name should be:
\begin{lstlisting}[frame=trBL]
public String getStudentName(){ 
    return studentName;
}
\end{lstlisting}

\textbf{Your Task:} Define a getter for all of the attributes of your class
using this general form.\\


\textit{Hint: Defining getters and setters is tedious so most modern IDEs can
auto-generate getters and setter. Explore your IDE and see if you can figure
out how to use this functionality.}

\subsubsection{String Overrides}

Finally, we want the ability to print the contents of our class in a pretty,
readable manner. By default, if you instantiate a class and then print it 
you will get the something along the lines of the following output:\\
\vspace{0.2cm}\\

\begin{framed}
\begin{minipage}{0.25\textwidth}
\underline{\textbf{Foo.java}}
\begin{lstlisting}
public class Foo{
   //...
}
\end{lstlisting}
\end{minipage}
\hfill
\vline
\hfill
\begin{minipage}{0.32\textwidth}
\underline{\textbf{Main.java}}
\begin{lstlisting}
public class Main{
   Foo c1 = new Foo();
   System.out.println(c1);
}
\end{lstlisting}
\end{minipage}
\hfill
\vline
\hfill
\begin{minipage}{0.28\textwidth}
\underline{\textbf{Terminal:}}
\begin{shell}
> java Main
Foo@6504e3b2
\end{shell}
\end{minipage}
\end{framed}
\vspace{0.2cm}

By overriding the \lstinline|toString| function we can define how this class
should be converted to a string. This in turn defines what string should be
printed when printing an instance of the class. For instance, if the
\lstinline|Foo| class has an attribute \lstinline|name| we can provide a 
\lstinline|toString| method for that class and our program will now run
in the following way:\\
\vspace{0.2cm}\\

\begin{framed}
\begin{minipage}{0.52\textwidth}
\textbf{Foo.java}
\begin{lstlisting}
public class Foo{
    public String name = "bar";
    //...
    @Override 
    public String toString(){
        return "Foo name: " + this.name;
    }
}
\end{lstlisting}
\end{minipage}
\hfill
\vline
\hfill
\begin{minipage}{0.42\textwidth}
\underline{\textbf{Main.java}}
\begin{lstlisting}
public class Main{
    Foo c1 = new Foo();
    System.out.println(c1);
}
\end{lstlisting}
\hline
\vspace{0.2cm}
\underline{\textbf{Terminal:}}
\begin{shell}
> java Main
Foo name: bar
\end{shell}
\end{minipage}
\end{framed}
\vspace{0.2cm}


\textbf{Your Task: } Override the \lstinline|toString| method such that when an instance of the \lstinline|Vehicle| class is printed it appears in the following form:
\begin{shell}
--Year Make Model--
Cylinders: CylinderCount
Drive: DriveConfig
Fuel Type: FuelType 
Transmission: TransmissionType
Class: VehicleClass

\end{shell}

\subsubsection{Comparable Override}

In your provided starter files you will notice that the constructor is declared in the following way:
\begin{lstlisting}[frame=trBL]
public class Vehicle implements Comparable<Vehicle>{
    //...
}
\end{lstlisting}
\lstinline|Comparable| is an interface. We will cover interfaces in greater
detail later but, for the time being it is sufficient to know that an interface
is a construct in Java which contains function headers which have not
been implemented. For \lstinline|Comparable|, one of these functions is the
\lstinline|compareTo| method which is used by functions such as
\lstinline|Collections.sort|. In a later portion of the assignment we will be
sorting collections of \lstinline|Vehicle| instances and, as such, we must
implement the \lstinline|compareTo| method by returning a comparison between
attributes in the class for which the \lstinline|compareTo| method has already
been implemented (e.g., \lstinline|String|, \lstinline|Integer|). For example:
\begin{lstlisting}[frame=trBL]
public class Foo implements Comparable<Foo>{
    String name;
    public Foo(String name){
        this.name = name
    }
    @Override
    public int compareTo(Foo other) {
        return name.compareTo(other.name);
    }
}
\end{lstlisting}


\textbf{Your Task:} Change the class such that it implements \lstinline|Comparable<Vehicle>| and implement the method \lstinline|compareTo| method such that it compares vehicle objects based on their year.

\subsubsection{Vehicle Class - Checklist}
\begin{itemize}
    \item Class Attributes:
        \begin{todolist}
            \item All needed attributes are declared.
            \item Attributess are \lstinline|private|.
            \item Attributess are \lstinline|final|.
        \end{todolist}
    \item Getters
        \begin{todolist}
            \item A getter function exists for each attribute.
            \item All getters are of the form \lstinline|getAttributeName|.
        \end{todolist}
    \item Overrides
        \begin{todolist}
            \item The \lstinline|tosString| method is present and returns a string of the specified format.
            \item The \lstinline|compareTo| method is present and returns the comparision between the intance and a \lstinline|Vehicle| instance that is passed in as a parameter. The comparison should be based on year of the vehicles being compared.
        \end{todolist}
\end{itemize}

