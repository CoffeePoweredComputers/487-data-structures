The XML document we are given contains many many fields with various
information on each type of vehicle. We want our parse to do the following:
\begin{enumerate}
  \item Parse the data in using the XML java DOM parser.
  \item Filter out the data we are interested in and retrieve the values.
  \item Instantiate Vehicle objects with those values.
\end{enumerate}


\subsubsection{Attributes}

For our class we will need two \lstinline|private| attributes:
\begin{enumerate}
    \item \lstinline|vehicleDatabase|  \textrightarrow \  An \lstinline|List| of \lstinline|Vehicle| instances. This will contain all of the vehicles in our database.
    \item \lstinline|makes| \textrightarrow \ An \lstinline|List| of strings. This will contain a list of all the unique makes in our database.
\end{enumerate}
You do not need to do anything for this task as those attributes are declared outside of the construction
and initialized within it.

\subsubsection{Constructor and XML Parsing}

For this portion of the assignment, create a constructor that takes a single
string, \lstinline|filepath|. This file path is intended to be either an
absolute or relative path to an XML file of the same form as vehicles.xml. The
purpose of our constructor is to parse this file and use it's contents to
populate our class's attributes.

The general template for parsing an XML file in Java using \lstinline|javax| is as 
follows:
\begin{lstlisting}[frame=trBL]
DocumentBuilderFactor dbf = DocumentBuilderFactory.newInstance();

DocumentBuilder db = null;
try{
    db = dbf.newDocumentBuilder();
} catch(ParserConfigurationException e{
    //...
}

File f = new File(filepath);

Document doc = null;
try{
    doc = db.parse(f);
} catch(SAXException | IOException e){
    //...
}
\end{lstlisting}

Following this set of operations you will be left with \lstinline|dbf|. From there
you should use the \lstinline|getElementsByTagName| function. This will produce 
a \lstinline|NodeList| which contains instances of the \lstinline|Element| class
where each \lstinline|Element| is a portion of XML that had the given tag. You
can then iterate over that \lstinline|NodeList| in the following way:

\begin{lstlisting}[frame=trBL, basicstyle=\fontnotesize]
NodeList nodes = doc.getElementsByTagName("foo")
for(int i = 0; i < nodes.getLength(); i++){
    Element elem = (Element) nodes.get(i);
    //process elem
}
\end{lstlisting}

\newpage
Use this to parse the document in the following way:
\begin{enumerate}
    \item Get a \lstinline|NodeList| of all tags in the XML document with the \lstinline|"vehicle"| tag.
    \item Iterate over the list and, at each iteration use to following structure to extract each vehicle attribute we are interested in:
        \begin{lstlisting}
        elem.getElementsByTagName(??).item(0).getTextContent();
        \end{lstlisting}
        Be sure to convert the text content to \lstinline|Integer| when needed.
    \item At each iteration create a new vehicle instance and add that instance to the \lstinline|vdb| list.
\end{enumerate}

\textbf{Your Task:} Reference section 2.1 to determine what the names for each
of the fields is. Go through the provided skeleton code and replace all
instances of ``??'' within the \lstinline|getElementByTagName| method with the
appropriate tag. Additionally, after extracting all the needed values,
instantiate a new vehicle class with those values and add it to the database.

\subsubsection{Getters}

The only getter you will need for this class is for the \lstinline|makes|
attribute. We will be building other query function that will allow the user to
interact with the database. This getter (\lstinline|getMakes|) has been provided for you.

\subsubsection{Query Functions}

To interact with the ``database'' we will be implementing the following functions 

\paragraph{\lstinline|public List<String> queryClasses(String
userMake)\{ ... \}|} This function should  search the database, and return a
\textit{sorted} List of string containing all the \textit{unique} vehicle
classes that are of the specified \lstinline|userMake|. Be sure to ignore the
case of the string \lstinline|userMake| when making comparisons.

\paragraph{\lstinline|public List<String> queryModels(String userMake,
String user_class)\{ ... \}|} This function should search the database and return a
\lstinline{sorted} List of strings containing all the \textit{unique}
vehicle models present in the database that have the specified
\lstinline|userMake| and \lstinline|userModel|. Be sure to ignore the case of
the parameters when making comparisons.

\paragraph{\lstinline|public List<Vehicle> queryVehicles(String
userModel)\{ ... \}|} Search the database and return an List of all the
\lstinline|Vehicle| instances in the database that match the model specified in
the parameter \lstinline|userModel|. Be sure to ignore the case of
\lstinline|userModel| when making comparisons.

\subsubsection{VehicleDatabase Class - Checklist}
\begin{itemize}
    \item Constructor: 
        \begin{todolist}
            \item Populates the \lstinline|vdb| and \lstinline|makes| attributes with data from the XML file.
        \end{todolist}
    \item Functions to Implement: 
        \begin{todolist}
            \item \lstinline|queryClasses|
            \item \lstinline|queryModels|
            \item \lstinline|queryVehicles|
        \end{todolist}
\end{itemize}

