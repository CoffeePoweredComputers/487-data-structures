As the name suggests one of the primary methods  of a Binary \textit{Search}
Tree is to provide the ability to \textit{search} for an access a specific node
in the tree. Searching will also form the basis for adding and removing nodes.
In a BST you should keep in mind the following rules:
\begin{enumerate}
    \item For a given root node, the left node is less than the root and the right node is greater than the root.
    \item The leftmost node in a tree contains the smallest value.
    \item The rightmost node in a tree contains the greatest value.
\end{enumerate}
The first of these rules is the one with which we will concern ourselves with
for performing a general search. The latter two can be used to search for the
greatest node or the least node in a tree.\\

\textbf{Your Task: } Using these rules, implement the following methods that
search for nodes in the BST.  These methods can be implemented either
recursively or iteratively. Whichever solution you choose to pursue, do take
care to use the proper method signature as detailed below:
\begin{itemize}
    \item A search method:
    \begin{itemize}
        \item \textbf{Iterative:} \lstinline|public TreeNode<E> search(E data){ ... }|
        \item \textbf{Recursive:} \lstinline|public TreeNode<E> search(TreeNode<E> curr, T data){ ... }|
    \end{itemize}
    \item A method that retrieves the minimum node in a tree:
    \begin{itemize}
        \item \textbf{Iterative:} \lstinline|public TreeNode<E> getMinimum(){ ... }|
        \item \textbf{Recursive:} \lstinline|public TreeNode<E> getMinimum(TreeNode<E> curr){ ... }|
    \end{itemize}
    \item A method that retrieves the maximum node in a tree:
    \begin{itemize}
        \item \textbf{Iterative:} \lstinline|public TreeNode<E> getMaximum(){ ... }|
        \item \textbf{Recursive:} \lstinline|public TreeNode<E> getMaximum(TreeNode<E> curr){ ... }|
    \end{itemize}
\end{itemize}
