\vspace{0.25cm}\\
\begin{minipage}{0.32\textwidth}
    \centering
    \vspace{0.1cm}
    \begin{figure}[H]
    \begin{tikzpicture}[level distance=1.5cm,
        level 1/.style={sibling distance=3cm},
        level 2/.style={sibling distance=1.5cm},
        every node/.style = {minimum width = 2em, draw, circle}
        ]
        \node {5}
            child {node {3}
                child {node {1}}
                child {node[fill=!10!orange] {4}}
            }
            child {node {6}
                child {edge from parent[draw = none]}
                child {node {8}}
            };
    \end{tikzpicture}
    \caption{Leaf}
    \label{fig:nodeleaf}
    \end{figure}
\end{minipage}
\hfill
\begin{minipage}{0.32\textwidth}
    \centering
    \begin{figure}[H]
    \begin{tikzpicture}[level distance=1.5cm,
        level 1/.style={sibling distance=3cm},
        level 2/.style={sibling distance=1.5cm},
        every node/.style = {minimum width = 2em, draw, circle}
        ]
        \node {5}
            child {node {3}
                child {node {1}}
                child {node {4}}
            }
            child {node[fill=!10!orange] {6}
                child {edge from parent[draw = none]}
                child {node {8}}
            };
    \end{tikzpicture}
    \caption{One Subtree}
    \label{fig:nodeonechild}
    \end{figure}
\end{minipage}
\hfill
\begin{minipage}{0.32\textwidth}
    \centering
    \begin{figure}[H]
    \begin{tikzpicture}[level distance=1.5cm,
        level 1/.style={sibling distance=3cm},
        level 2/.style={sibling distance=1.5cm},
        every node/.style = {minimum width = 2em, draw, circle}
        ]
        \node {5}
            child {node[fill=!10!orange] {3}
                child {node {1}}
                child {node {4}}
            }
            child {node {6}
                child {edge from parent[draw = none]}
                child {node {8}}
            };
    \end{tikzpicture}
    \caption{Two Subtrees}
    \label{fig:nodetwochildren}
    \end{figure}
\end{minipage}
\vspace{0.25cm}\\
Removing a node once again begins with a search, however, once we have found
the node we wish to remove, there are three situations we must consider:
\begin{enumerate}
    \item The node we are removing is a leaf (Figure~\ref{fig:nodeleaf}).
    \item The node we are removing has one subtree(Figure~\ref{fig:nodeonechild}).
    \item The node we are removing has two subtrees (Figure~\ref{fig:nodetwochildren}).
\end{enumerate}

\textbf{Your Task: } Your task will be to create the following method:
\lstinline|public void remove(??){ ... }|. The parameters of this method
 are up to you and the approach you take. A suggested strategy for dealing with
 the various removal cases enumerated above is detailed in the following
 sections.

