{\renewcommand{\arraystretch}{1.2}% for the vertical padding}
\begin{tabular}{| c | l |}
    \hline
    Key & Value \\
    \hline
    A & \rule[11.2ex]{0pt}{0pt}  \begin{tikzpicture}[list/.style={rectangle split, rectangle split parts=3, draw}, start chain]

            \foreach \x/ \w in {D/5}{
                \node[list,on chain] (\x) {dest=\x \nodepart{second} weight=\w};
            }


            \node[on chain,draw,inner sep=6pt] (N) {};
            \draw (N.north east) -- (N.south west);
            \draw (N.north west) -- (N.south east);


            \foreach \s/ \d in {D/N}{
                \draw[*->] let \p1 = (\s.three), \p2 = (\s.three) in (\x1,\y2) -- (\d);
            }

    \end{tikzpicture}
    \\ \hline
    B & \rule[11.2ex]{0pt}{0pt} \begin{tikzpicture}[list/.style={rectangle split, rectangle split parts=3, draw}, start chain]

            \foreach \x/ \w in {A/7}{
                \node[list,on chain] (\x) {dest=\x \nodepart{second} weight=\w};
            }


            \node[on chain,draw,inner sep=6pt] (N) {};
            \draw (N.north east) -- (N.south west);
            \draw (N.north west) -- (N.south east);


            \foreach \s/ \d in {A/N}{
                \draw[*->] let \p1 = (\s.three), \p2 = (\s.three) in (\x1,\y2) -- (\d);
            }

    \end{tikzpicture}
    \\ \hline
    C & \rule[11.2ex]{0pt}{0pt} \begin{tikzpicture}[list/.style={rectangle split, rectangle split parts=3, draw}, start chain]

            \foreach \x/ \w in {B/8}{
                \node[list,on chain] (\x) {dest=\x \nodepart{second} weight=\w};
            }


            \node[on chain,draw,inner sep=6pt] (N) {};
            \draw (N.north east) -- (N.south west);
            \draw (N.north west) -- (N.south east);


            \foreach \s/ \d in {B/N}{
                \draw[*->] let \p1 = (\s.three), \p2 = (\s.three) in (\x1,\y2) -- (\d);
            }
    \end{tikzpicture}
    \\ \hline
    D & \rule[11.2ex]{0pt}{0pt} \begin{tikzpicture}[list/.style={rectangle split, rectangle split parts=3, draw}, start chain]
            \foreach \x/ \w in {B/1}{
                \node[list,on chain] (\x) {dest=\x \nodepart{second} weight=\w};
            }


            \node[on chain,draw,inner sep=6pt] (N) {};
            \draw (N.north east) -- (N.south west);
            \draw (N.north west) -- (N.south east);


            \foreach \s/ \d in {B/N}{
                \draw[*->] let \p1 = (\s.three), \p2 = (\s.three) in (\x1,\y2) -- (\d);
            }


    \end{tikzpicture}
    \\ \hline
    E & \rule[11.2ex]{0pt}{0pt} \begin{tikzpicture}[list/.style={rectangle split, rectangle split parts=3, draw}, start chain]

            \foreach \x/ \w in {D/15, B/7, C/5}{
                \node[list,on chain] (\x) {dest=\x \nodepart{second} weight=\w};
            }


            \node[on chain,draw,inner sep=6pt] (N) {};
            \draw (N.north east) -- (N.south west);
            \draw (N.north west) -- (N.south east);


            \foreach \s/ \d in {D/B, B/C, C/N}{
                \draw[*->] let \p1 = (\s.three), \p2 = (\s.three) in (\x1,\y2) -- (\d);
            }



    \end{tikzpicture}
    \\ \hline
    Z & \rule[11.2ex]{0pt}{0pt} \begin{tikzpicture}[list/.style={rectangle split, rectangle split parts=3, draw}, start chain]

    \end{tikzpicture}
    \\ \hline

\end{tabular}
}

